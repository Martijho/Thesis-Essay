\chapter{Introduction}
From essay. 
More on multi task learning
More on transfer learning

\section{Raise problem: catastrophic forgetting.}
Multiple solutions (PNN, PN, EWC)
- Large structures (PNN, PN)
- Limited in number of tasks it can retains(EWC)

Optimize reuse of knowledge while still providing valid solutions to tasks. More reuse and limited capacity use will increase amount of task a structure can learn. 

where do i start?
Question DeepMind left unanswered is how different GAs influence task learning and module reuse. 
Exploration vs exploitation\ref{theoretic background on topic}

why this? broad answers first, specify later. 
We know PN works. would it work better for different algorithms?
logical next step from original paper "unit of evolution"

\section{Problem/hypothesis}
* What do modular PN training do with the knowledge? 
- More/less accuracy?
- More/less transferability? 
Test by learning in end-to-end first then PN search. 
Difference in performance or reuse?

* Can we make reuse easier by shifting focus of search algorithm?
- PN original: Naive search. Higher exploitation improve on module selection?

\section{How to answer?}
- Set up simple multitask scenarios and try. 
* 2 tasks where first are end to end vs PN
* List algorithms with different selection pressure and try on multiple tasks. 

\iffalse
    What is the use of a Nifty Gadget? 
    What is the problem? 
    How can it be solved? 
    What are the previous approaches? 
    What is your approach? 
    Why do it this way? 
    What are your results? 
    Why is this better? 
    Is this a new approach? 
    Why haven't anyone done it before? 
    or
    Why do you reiterate previous work? 
    What is your contribution to the field of Nifty Gadgets? 
    
    \section{What should this chapter contain?}
    Presentation of the problem or phenomenon to be addressed, the situation where the problem or phenomenon occurs, and references to earlier relevant research. 
    \subsection{Common errors}
    Problem is not properly specified or formulated; insufficient references to earlier work.  
    
    \section{Purpose}
    What can be gained by more knowledge about the problem or phenomenon. 
    \subsection{Common errors}
    The purpose is not mentioned, not connected to earlier research, or not in line with what the actual contents of the thesis.  
    
    \section{Problem/Hypothesis} 
    Questions that need to be answered to reach 
    the goal and/or hypothesis formulated be means of 
    underlying theories. 
    \subsection{Common errors}
    Missing problem description; deficiencies in the connections between questions; badly formulated 
    hypothesis.  
    
    \section{Method} 
    Choice of an adequate method with respect to the 
    purpose and problem/hypothesis. 
    
    \subsection{Common errors}
    An inappropriate method is used, for example due to lack of knowledge about different methods; 
    erroneous use of chosen method.  
\fi