\chapter{Implementation}
% EDIT NOTE: 
% Limit overlap in implementation details between this chapter and 
% experimentation implementation. Build up a base that can be built on in chapter
% 4 and 5. 
\section{Python implementation}
- why python? 
    - Problems: 
        - Not quick to run 
    - Pros: 
        - Quick to prototype in 
        - Generally good to debug
        - Multiple good tools for machine learning
            - \cite{tensorflow}
            - \cite{keras}
            - Why are these good?
        - Other packages
            - Matplotlib (visualization)
            - Numpy (math stuffs)
            - Pickle (data logging)
- code structure
    - Object oriented
        - Easily parameterizable for ease of prototyping pathnet structures
    - Class structure: 
        - Modules
        - Layers
        - PathNet
            - Functionality for
                - Building random paths
                - Creating keras models
                - static methods for creating pathnet structures
                - reset backend session
        - Taks
        - Search
        - Plot generating
- Training on gpu
    - Quicker in general for ML
    - This implementation do lots on CPU
        - Other implementations could take advantage of customizing layers and models in keras. 
- Noteable differences in implementation 
    - Keras implementasjon
    - Path fitness not negative error but accuracy
    - exp 2: fitness calculated before evaluation (not same step)
    - Not added any noise to training data
- Implementation problems
    - Tensorflow sessions not made for using multiple graphs
        - Resetting backend session after a number of models are made
    - Tensorflow-gpus default is using all gpu memory it can 
        - Limiting data allocation to scale when needed
    - Tensorflow session does not free allocated memory before python thread is done. 
        - Run all experiments through treads. 
- Code available on github

\section{datasets}
- Data type
- Use cases and citations
- How does the data look?
- set sizes and class distributions
- state of the art and human level performance
\cite{MNIST}
\cite{SVHN}

\section{Search implementation}
- functions. callback to theoretical background and GA buzzwords
- parameterization


\iffalse
    X   Can you describe your implementation in detail? 
    X   Why did you use this technology? 
    X   How does the theory relate to your implementation? 
    X   What are your underlying assumptions? 
    X   What did you neglect and what simplifications have you made? 
    X   What tools and methods did you use? 
    X   Why use these tools and methods? 
\fi