\begin{table}[h!]
    \centering
    \begin{tabular}{cccccccc}
                  & \textbf{1a}  & \textbf{1b}  & \textbf{1c}  & \textbf{2a}  & \textbf{2b}  & \textbf{3a}  & \textbf{3b}  \\
    \textbf{1b}   & 0.72907 &         &         &         &         &         &              \\
    \textbf{1c}   & 0.13284 & 0.30542 &         &         &         &         &              \\
    \textbf{2a}   & 0.77366 & 0.58060 & 0.07826 &         &         &         &              \\
    \textbf{2b}   & 0.20366 & 0.46155 & 0.87357 & 0.20150 &         &         &              \\
    \textbf{3a}   & 0.21488 & 0.44626 & 0.96039 & 0.18693 & 1.00000 &         &              \\
    \textbf{3b}   & 0.60388 & 0.89446 & 0.45577 & 0.50363 & 0.59275 & 0.54930 &              \\
    \textbf{Rand} & 0.00040 & 0.00034 & 0.00000 & 0.00155 & 0.00014 & 0.00003 & 0.00042
    \end{tabular}
    \caption[Experiment 3: p-table for reuse]{Two-sided p-values for the distribution of module ruse where each group is the joined reuse for all generation limits for one algorithm. \(\alpha\) value 0.00179. The p-value for the comparison of 1b and the estimated reuse with random module selection is on the order of \(10^{-6}\), which made it round to 0 for this table.}
    \label{tab:exp3.reuseptable}
\end{table}
